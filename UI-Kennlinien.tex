\documentclass[12pt,a4paper,ngerman]{report}
\usepackage{babel}
%\usepackage{natbib}
\usepackage{url}
%\usepackage[left=2cm, right=1.5cm, top=2cm, bottom=2cm]{geometry}
%\usepackage[ansinew]{inputenc}
\usepackage{amsmath}
\usepackage{nicefrac} % macht schöne Brüche mit querstrich mit \nicefrac{1}{2}
\usepackage{graphicx}
%\graphicspath{}
\usepackage{titlesec}% um chapterüberschriften anzupassen.
\titleformat{\chapter}{\normalfont\huge\bf}{\thechapter.}{20pt}{\huge\bf}
\usepackage{parskip}
\usepackage{fancyhdr}
\usepackage{amsfonts}
\usepackage{float}
\usepackage{caption}
\usepackage{subcaption} % for \begin{subfigure}
	
	\usepackage{csquotes} % mit \enquote{blabla} tolle anfürungsstriche erstellen
	%\usepackage{physics} %lässt mich \bra und \ket benuzen %im konflict mit siunitx
	
	\usepackage{pgfplots} %für plots
	\pgfplotsset{compat=newest}
	
	\usepackage{varioref} % macht mit \vref{} viel bessere referenzen
	\usepackage{hyperref} % macht klickbare referenzen
	
	\usepackage{xcolor, soul} %mit \hl{} kann man toll Sachen hervorheben.
	\newcommand{\highlight}[1]{%
		\colorbox{yellow!50}{$\displaystyle#1$}} % mit \highlight{} kann man sogar in Gleichungen hervorheben
	
	\usepackage{vmargin}
	\usepackage[section]{placeins}
	\usepackage{capt-of}
	\usepackage{enumitem}
	\usepackage{multirow}
	\usepackage{blindtext}
	\usepackage[version=4]{mhchem} % um Chemische Elementsymbole zu benutzen: \ce{H20}
	
	\usepackage{pdfpages} % um PDFs einzufügen
	
	%spread to latex:
	\usepackage{booktabs, multirow} % for borders and merged ranges
	\usepackage{changepage,threeparttable} % for wide tables
	
	\providecommand{\e}[1]{\ensuremath{\cdot 10^{#1}}}
	\providecommand{\fehlt}{\textcolor{red}{\emph{Fehlt!\dots}}}
	
	\usepackage{siunitx}
	\sisetup{
		locale = DE ,
		separate-uncertainty = false,
		%per-mode = fraction,
		%per-mode = symbol
	}
	\DeclareSIUnit\bar{bar}
	\DeclareSIUnit\atomicmassunit{u}
	
	
	
	\setmarginsrb{3 cm}{2.5 cm}{3 cm}{2.5 cm}{1 cm}{1.5 cm}{1 cm}{1.5 cm}
	\title{UI-Kennlinien}			%%%%%%%%%%
	% Title
	
	
	\author{Frederik Uhlemann, F. Adamczyk}
	% Author
	\date{\today}
	% Date
	
	\makeatletter
	\let\thetitle\@title
	\let\theauthor\@author
	\let\thedate\@date
	\makeatother
	
	\pagestyle{fancy}
	\fancyhf{}
	\rhead{\theauthor}
	\lhead{UI-Kennlinien}
	\cfoot{\thepage}
	%%%%%%%%%%%%%%%%%%%%%%%%%%%%%%%%%%%%%%%%%%%%
	\begin{document}
		
		%%%%%%%%%%%%%%%%%%%%%%%%%%%%%%%%%%%%%%%%%%%%%%%%%%%%%%%%%%%%%%%%%%%%%%%%%%%%%%%%%%%%%%%%%
		
		\begin{titlepage}
			\centering
			\vspace*{0.5 cm}
			% \begin{large} Justus-Liebig-Universität\\ Gießen \end{large}
			\includegraphics[width = 0.6 \textwidth]{JLU_Giessen-Logo}	%University Logo
			\\[2.0 cm]
			% \begin{center}    \textsc{\Large Justus - Liebig - Universität}\\{Giessen}\\[0.8cm]	\end{center}% University Name
			Versuch 4 des\\
			\textsc{\Large  Fortgeschrittenen-Praktikums}\\ [0.3 cm]				% Course Code
			\rule{\linewidth}{0.2 mm} \\[0.4 cm]
			{ \huge \bfseries \thetitle}\\%%% TITEL HERE
			\rule{\linewidth}{0.2 mm}\\
			Versuchstermin Freitag, 17.05.2024 \\
			~ \\
			[2.0 cm]
			
			
			\begin{minipage}{0.49\textwidth}
				\begin{flushleft}
					\emph{Praktikumsbetreuer:}\\
					Marius Müller\\
					%  Affiliation\\
					\small{\href{mailto:marius.mueller@physik.uni-giessen.de}{marius.mueller@physik.uni-giessen.de}}
				\end{flushleft}
			\end{minipage}~
			\begin{minipage}{0.49\textwidth}
				\begin{flushright}
					\emph{Protokoll von:} \\
					
					\large{Frederik Uhlemann}\\
					\small{\href{mailto:frederik-vincent.uhlemann@physik.uni-giessen.de}{frederik-vincent.uhlemann@physik.uni-giessen.de}\\~\\
						%Matrikel Nr.: \:  \\[0.5cm]
						%\href{mailto:}{}
					}
					\large{Florian Adamczyk} \\
					\small{\href{mailto:florian.marius.adamczyk@physik.uni-giessen.de}{florian.marius.adamczyk@physik.uni-giessen.de}\\
						%Matrikel Nr.: \: 8105234}
				}
			\end{flushright}
		\end{minipage}
		
	\end{titlepage}
	
	%%%%%%%%%%%%%%%%%%%%%%%%%%%%%%%%%%%%%%%%%%%%%%%%%%%%%%%%%%%%%%%%%%%%%%%%%%%%%%%%%%%%%%%%%
	\setcounter{secnumdepth}{3}
	\setcounter{tocdepth}{4}
	\tableofcontents
	%\newpage
	
	%%%%%%%%%%%%%%%%%%%%%%%%%%%%%%%%%%%%%%%%%%%%%%%%%%%%%%%%%%%%%%%%%%%%%%%%%%%%%%%%%%%%%%%%%
	%\renewcommand{\thesection}{\arabic{section}} %lässt in den subsections die erste zahl von darüberliegenden chapter weg.
	
	%\pagebreak
	
	%\setcounter{chapter}{-1}
	\chapter*{Einleitung}
	\addcontentsline{toc}{chapter}{Einleitung}
	\fehlt
	
	
	
	\chapter{Versuchsaufbau und Durchführung}
	\fehlt 
	
	\chapter{Auswertung} %seehr ausführlich!!! 80-90% des Protokolls!!!
	\fehlt
	
		\section{Diode}
		
		\section{Transistor}
		
		\section{Solarzelle}
	
	
	
	\chapter{Fazit}
	
	\listoffigures
	\addcontentsline{toc}{chapter}{\listfigurename}
	
	\begin{thebibliography}{111} \addcontentsline{toc}{chapter}{Literaturverzeichnis}
		\bibitem{Anleitung}
		I. Physikalisches Institut, \glqq Versuch 1.8: I-U-Kennlinien an Halbleitern
		und Solarzellen \grqq{}, 2024.
		
		\bibitem{beuth}
		K. Beuth, \glqq Elektronik 2 – Bauelemente\grqq, Vogel Buchverlag (Bauelemente)
		
		\bibitem{hunklinger}
		S. Hunklinger, \glqq Festkörperphysik\grqq, Oldenbourg Wissenschaftsverlag (Grundlagen)
		
		
		
	\end{thebibliography}
	
	
	\chapter*{Anhang} \label{ch:Anhang}
	\addcontentsline{toc}{chapter}{Anhang}
	\FloatBarrier
	
	
	
	
	
	
\end{document}
